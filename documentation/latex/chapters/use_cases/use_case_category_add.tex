\subsection{Setup a new alert category}
Define a category to sort alerts.
\subsubsection{Scope}
The ClockAlarm manager window.
\subsubsection{Primary actor}
User
\subsubsection{Precondition}
ClockAlarm is running. The configurations and existing alerts are loaded. The user is on the main window.
\subsubsection{Postcondition}
A new alert category is created. It defines the parameters of the alerts belonging to it.
\subsubsection{Main success scenario}
\begin{enumerate}
	\item The user browses the menu and selects ``Edit \textgreater~Categories''. The categories manager window is displayed.
	\item The user clicks on the ``New category'' button. The add category dialog is displayed.
	\item\label{itm:ucca_enter_sc}The user defines the new category. The name is mandatory. \\The user can define a font and a color for the alerts and the sound to be used.
	\item\label{itm:ucca_validate_sc} The user validates the category. The category is created
	\item The categories manager window is updated.
	\item The creation window closes.
\end{enumerate}
\subsubsection{Extension}
\begin{enumerate}
	\item[\ref{itm:ucca_validate_sc}] A non-mandatory field is not filled.
	\begin{enumerate}[i]
		\item The default settings are used. 
		\item The category is created.
		\item The categories manager window is updated and the creation window closes.
	\end{enumerate}
	
	\item[\ref{itm:ucca_validate_sc}] One of the parameter is invalid.
	\begin{enumerate}[i]
		\item The user is asked to check his entries. Back to point~\ref{itm:ucca_enter_sc}.
	\end{enumerate}
\end{enumerate}