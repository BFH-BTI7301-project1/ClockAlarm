\section{Word Processor}
In order to write this document, we had to choose between several word processors. We required that the file could be used with a version control system so that the documentation and code could live in the same place. Also, we didn't want to loose time if there was a need to change the layout. After discussion, we came up with the following list:
\begin{enumerate}
\item \Gls{microsoft_word}
\item \Gls{google_docs}
\item \Gls{libre_office}
\item \Gls{scribus}
\item \Gls{tex}
\end{enumerate}
The main problem with number 1 and 3 is that the output of these programs are binary files. A built in versioning functionality exits in \gls{microsoft_word} but it adds another version control layer to the workflow. The .docx and .odt file types are in fact ZIP archives containing XML documents which can be uncompressed. They could be stored in the uncompressed state to be compatible with the chosen version control system \cite{zipdocextension}. Or they could be converted into another format. Clearly this is not convenient. Therefore we decided that they were not compatible with our first requirement stated above.\\
Using a Google Doc means that the document will be stored on Google's servers and not within the folder containing the source code. I thus also violates our first requirement.\\
\Gls{scribus} is a free alternative to \gls{indesign} and doesn't produce binary files. It allows a fine-grained control of the layout, frames and styles \cite{ibm2013open}.\\
Finally there is \gls{tex}, a typesetting system created by Knuth. It provides a low-level language that is not directly used when writing documents. Instead there exist higher level formats such as \gls{latex} or Plain \TeX (much more lower level than \gls{latex}) which provide a large set of macros \cite{levels2017}.\\
Our supervisor strongly recommended us the use of \gls{latex}, that is why we ended up using it.