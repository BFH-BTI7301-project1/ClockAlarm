\subsection{Launch ClockAlarm manager (no login)}
In the case of an application without shared database and user accounts.
\subsubsection{Scope}
A computer running any common operating system (Windows, Mac OS, Linux).
\subsubsection{Primary actor}
User or Administrator
\subsubsection{Precondition}
The computer is on and the user is logged on his computer user session. The ClockAlarm application is correctly installed.
\subsubsection{Postcondition}
ClockAlarm is running and the manager window is open. The user is logged on his personnal ClockAlarm session.
\subsubsection{Main success scenario}
\begin{enumerate}
	\item The user launches the ClockAlarm application.
	\item \label{itm:start} The application starts and loads configurations and alarms from the user's personal files folder.
	\item \label{itm:winopen} The manager window opens and displays the main window.
\end{enumerate}
\subsubsection{Extension}
\begin{enumerate}
	\item[\ref{itm:start}] The application il allready running in background.
	\begin{enumerate}[i]
		\item Goto point \ref{itm:winopen}.
	\end{enumerate}
	
	\item[\ref{itm:start}] Configurations or alerts can not be loaded (e.g. first use of the program).
	\begin{enumerate}[i]
		\item The application creates a new default configuration file and an empty alert file.
		\item Goto point \ref{itm:start}.
	\end{enumerate}
\end{enumerate}

\subsection{Launch ClockAlarm manager (with login)}
In the case of an application with a shared database and user accounts.
\\ \textbf{This solution is unlikely to be kept.}
\\For this reason, not all scenarios, especially those related to the connection with the server, will be treated here.
\subsubsection{Scope}
A computer running any common operating system (Windows, Mac OS, Linux).
\subsubsection{Primary actor}
User or Administrator
\subsubsection{Precondition}
The computer is on and the user is logged on his computer user session. The ClockAlarm application is correctly installed.
\\ The user has a registered user account on the server.
\subsubsection{Postcondition}
ClockAlarm is running and the manager window is open. The user is logged on his personnal ClockAlarm session.
\subsubsection{Main success scenario}
\begin{enumerate}
	\item The user launches the ClockAlarm application.
	\item \label{itm:start} The application starts. 
	\item \label{itm:check} The user is asked to enter his credentials and the program tries to check these.
	\item The application is connected to the server.
	\item \label{itm:load} The application loads configurations and alarms from the server database.
	\item \label{itm:winopen} The manager window opens and displays the main window.
\end{enumerate}
\subsubsection{Extension}
\begin{enumerate}
	\item[\ref{itm:start}] The application il allready running in background.
	\begin{enumerate}[i]
		\item Goto point \ref{itm:winopen}.
	\end{enumerate}
	
	\item[\ref{itm:check}] The credentials are incorrect.
	\begin{enumerate}[i]
		\item The connection to the server is denied.
		\item The user is asked to give his credentials again.
		\item Goto point \ref{itm:check}.
	\end{enumerate}
	
		\item[\ref{itm:check}] The user isn't registered.
	\begin{enumerate}[i]
		\item The connection to the server is denied.
		\item The user is redirected to a page to an account creation page.
		\item The user creates a new account on the server.
		\item Goto point \ref{itm:check}.
	\end{enumerate}
	
	\item[\ref{itm:load}] Configurations or alerts can not be loaded (e.g. first use of the account).
	\begin{enumerate}[i]
		\item The application creates and upload a new default configuration file and an empty alert file.
		\item Goto point \ref{itm:start}.
	\end{enumerate}
\end{enumerate}