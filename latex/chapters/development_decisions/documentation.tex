\section{Documentation}\label{sec:documentation}

For documenting our project we use Sphinx%
\footnote{\url{http://www.sphinx-doc.org}} combined with Read
The Docs\footnote{\url{https://readthedocs.org}} to host the
documentation. Read The Docs builds the documentation every time there is a new
commit (A git hook has to be set up beforehand for it to work).

We choose Sphinx because it is the most popular tool used in the python
community and it is officially recommended (The official Python documentation is
created using Sphinx).

During development we realised thanks to Sphinx that our project had cycles. See
Section~\ref{sec:cycles} for additional details.

After retrospection, this document could have been written using Sphinx. It has
the same capabilities than \LaTeX{} and can generate the document in multiple
formats: pdf, html, etc..
