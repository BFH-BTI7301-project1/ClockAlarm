\section{Unit Testing}

Python comes with a couple of built-in modules for testing the source code:

\begin{enumerate}
    \item Unittest
    \item Doctest
    \item Unittest.mock (Only for Python 3.3)
\end{enumerate}

The Unittest framework is similar to the JUnit one: ``It supports test
automation, sharing of setup and shutdown code for tests, aggregation of tests
into collections, and independence of the tests from the reporting
framework.''~\cite{pythondoc361unittest}

Doctests are put in comments and are formatted in form of an interactive Python
session. They are usually very simple and give informative examples of the usage
of the class or function to the reader.~\cite{python361doctest}

Unittest.mock allows to replace parts of the system under test with mock objects
and make assertions about how they have been used.~\cite{pythondoc370mock}
\\

Other alternatives include: pytest, nose2 and tox.

Pytest makes it easy to write tests: only plain assert statements are used,
which makes the code look very clean and neat. It can run unittest and nose test
suites without any additional configuration nor modification.~\cite{pytest}

Nose2 is the successor of nose. Its goals are to extend unittest and make
testing easier to understand.~\cite{nose2VSnose}

Tox is a test framework and virtualenv management tool that allows to run tests
in multiple environments.~\cite{tox270doc}\\

We chose to use pytest because as already mentioned, the code is cleaner and
more readable. Less work is also required to achieve the same result as with the
other solutions. We don't use virtual environments so tox is of no use to us.
