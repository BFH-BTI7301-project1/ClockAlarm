\section{Database}

We planned to have three alert types:

\begin{enumerate}
    \item Simple non recurring text alerts
    \item Recurring simple alerts
    \item Alerts that send emails
\end{enumerate}

All three types are independent of each other, that is, there is no logical link
between them. The only ``link'' that they have in common is that they are
extending the abstract class \texttt{Alert}. Therefore we don't need a
relational database, only a very lightweight one.

The requirement for the file storing the database is that it should be humanly
readable, i.e., it should be in a non binary format. We came up with four
possibilities for the file format:

\begin{itemize}
    \item XML
    \item JSON
    \item YAML
    \item Plaintext
\end{itemize}

BaseX\footnote{For python it is the BaseXCLient:
\url{http://basex.org}} handles both JSON and XML databases. Another
alternative is TinyDB
\footnote{\url{http://tinydb.readthedocs.io}}, it
handles only databases in the JSON format. PyXAML
\footnote{\url{http://pyyaml.org}} handles databases stored in the
YAML
format. And for a plaintext database, i.e., a simple text file, everything has
to be handled with the Python \texttt{open(\ldots)} and \texttt{write(\ldots)}
commands.

We choose to use TinyDB and the JSON file format because our supervisor
encouraged us to do so and because the syntax is very simple and follows the
Pythonic way.
