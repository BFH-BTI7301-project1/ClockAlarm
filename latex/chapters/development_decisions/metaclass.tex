\section{Metaclass}

A \gls{metaclass} issue was discovered when we tried to apply the new python 3 style
for abstract classes. In our case, the \texttt{Alert} class is an abstract class
that extends \texttt{QObject}. As \texttt{QObject} has its metaclass, pyhton
could not resolve which metaclass \texttt{Alert} should inherit. Therefore a
special metaclass \texttt{AlertMeta} had to be created. It inherits
\texttt{QObject}'s metaclass and is abstract. The \texttt{Alert} class then
has the metaclass \texttt{AlertMeta} and extends
\texttt{QObject}.~\cite{sim2017meta,maries2015meta}

The difference is shown in Listings~\ref{metapy2} and~\ref{metapy3}.

\lstdefinestyle{custompy}{
  belowcaptionskip=1\baselineskip,
  breaklines=true,
  frame=l,
  xleftmargin=\parindent,
  language=Python,
  showstringspaces=false,
  basicstyle=\footnotesize\ttfamily,
  keywordstyle=\bfseries\color{green!40!black},
  commentstyle=\itshape\color{purple!40!black},
  identifierstyle=\color{blue},
  stringstyle=\color{orange},
}
\lstset{style=custompy,caption={Abstract class, Python 2.7},label=metapy2}
\begin{lstlisting}
...
class Alert(PyQt5.QtCore.QObject):
    __metaclass__ = abc.ABCMeta
    ...
\end{lstlisting}

\lstset{style=custompy,caption={Abstract class, Python 3.6},label=metapy3}
\begin{lstlisting}
...
class AlertMeta(type(PyQt5.QtCore.QObject), abc.ABC):
    pass

class Alert(pyqt.QObject, metaclass=AlertMeta):
    ...
\end{lstlisting}
