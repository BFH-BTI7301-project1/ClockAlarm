\section{Meeting~\arabic{section}: May 19, 2017}

\subsection*{Present at meeting}
Samuel Gauthier, Loïc Charrière, Claude Fuhrer

\subsection*{Agenda}
\begin{itemize}
    \item Show version with customizable notifications including sound, font,
        color selection and recurrence. Import / export notifications file
        functionality.
        (state: commit \href{https://github.com/BFH-BTI7301-project1/%
        ClockAlarm/tree/301822f044941c5058649d715016e683bc59eee8}{301822f})
\end{itemize}

\subsection*{Notes}

As we reach the end of the project we should freeze the functionalities as they
are now. The only modification that should be made is redesign the pop up
because it is ugly.

\textbf{Question:} Now when the user selects a custom sound it is copied into
the sound folder. If a sound file with the same name is present it is
overridden. Is this behavior ok?

\textbf{Answer:} Very reasonable behavior but we should document it.

The functionalities that we didn't have time to implement and the bugs should be
documented.

The Use Cases are too long, move them to the appendix. The documentation should
not exceed 30 to 40 pages. Otherwise, it is not read by the expert.

Document the basic Sphinx usage.

The docstrings should contain either the preconditions or the raised exceptions
of the function. The information put into the docstrings is information that
can't be found in the code.

Document the python requirements and how to launch the program.

\textbf{Question:} How deep should our explanations go during the final
presentation?

\textbf{Answer:} We should direct the presentation so that our class colleagues
can understand what we did. They do not know our project so for the very complex
parts of the code, show an expert or pseudo code.

\subsection*{Tasks}
    \begin{itemize}
        \item Write the tests and the appropriate documentation
        \item Rewrite the readme and add the installation instructions with
            python requirements.
    \end{itemize}
\subsection*{Next Meeting}
TBD

