\chapter{Conclusion}

As a reminder, the general requirements were that the application should be
cross-platform, the database stored in a non binary format, the configuration
and alerts easily transferable, the alerts customisable, classifiable into
categories, recurrent, delayable, snoozable. Moreover emails should be able to
be scheduled and sent at a given time.

We did not implement categories nor email that can be scheduled. The former
because of time constraints and the latter because it already exists as a plugin
in Mozilla Thunderbird and Outlook. Also we would have been dependent on those
two clients.

Looking back on choosing Python to develop this project we found out that with
this programming language, it was easy to quickly get to a reasonable result.
Many packages in the Python Package Index provide solutions to encountered
problems. Such as in our case, the sound problem\footnote{See
Section~\ref{sec:sound} for more information}. One drawback with so many
existing packages is that a lot are not actively maintained. This leads to
problems when using a newer version of Python.

Our biggest development challenge was Qt. We found it difficult to work with
because of its nature: slots and connections with signals have to be set up, two
subjects that are not easily understandable. Most of the time the documentation
for Python is not very helpful. Generally we were guided to the original C++
documentation which provided a far better help.

The actual state of the project allows users to extend the \texttt{Alert} class
so that new alert types can be implemented. In our view, adding categories would
be the next most important feature. Improving the console support and
providing a better installation process (setup.py) could lead this application
to get into the Python Package Index.

Our final impression is that it is clearly feasible to develop a better, free
and open source alternative to KAlarm that works on all major platforms.
