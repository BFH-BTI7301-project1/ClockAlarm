\section{Edit an alert category}\label{subsec:usecase_edit_category}

\subsubsection{Scope}
The ClockAlarm manager window.
\subsubsection{Primary actor}
User
\subsubsection{Precondition}
ClockAlarm is running. The configurations and existing alerts are loaded. The user is on the main window.
\\A category allready exist.
\subsubsection{Postcondition}
The selected category is updated as wanted. It defines the parameters of the alerts belonging to it.
\\ If alerts are defined by this category, they must be updated too.
\subsubsection{Main success scenario}
\begin{enumerate}
	\item The user browse the menu and selects ``Edit \textgreater~Categories''. The categories manager window is displayed. 
	\item The user selects from the list the category he wishes to modify.
	\item The user clicks on the ``Edit category'' button. The edit category dialog is displayed.
	\item\label{itm:eaac_enter_sc}The user update the selected category. The name is mandatory. \\The user can define or redifine a font and a color for the alerts and the sound to be used.
	\item\label{itm:eaac_validate_sc} The user validates the category. The category is updated, as well as all the alerts it defines.
	\item The categories manager window is updated.
	\item The creation window closes.
\end{enumerate}
\subsubsection{Extension}
\begin{enumerate}
	\item[\ref{itm:eaac_validate_sc}] A non-mandatory field is not filled.
	\begin{enumerate}[i]
		\item The initial setting of this parameter is used. 
		\item The category is updated.
		\item The categories manager window is updated and the dialog box closes.
	\end{enumerate}
	
	\item[\ref{itm:eaac_validate_sc}] One of the parameter is invalid.
	\begin{enumerate}[i]
		\item The user is asked to check his entries. Back to point~\ref{itm:eaac_enter_sc}.
	\end{enumerate}
\end{enumerate}
